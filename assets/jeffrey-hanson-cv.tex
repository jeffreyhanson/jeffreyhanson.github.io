\documentclass[12pt,a4paper]{article}
% load packages
\usepackage{amsmath,amsfonts,float,makecell,titletoc,tocloft,lineno,booktabs,subfiles,textcomp,blindtext,graphicx,natbib,url,hyperref}
\usepackage[explicit]{titlesec}
\usepackage[T1]{fontenc}
\usepackage[normalem]{ulem}
\usepackage{lmodern}
\usepackage[utf8]{inputenc}
\usepackage[onehalfspacing]{setspace}
\usepackage[english]{babel}
\usepackage{etoolbox}
\usepackage[left=0.75in,top=0.6in,right=0.75in,bottom=0.6in]{geometry}

% format links
\hypersetup{colorlinks=true,urlcolor=blue}

% list formatting
\usepackage{enumitem}
\setlist[description]{align=left,leftmargin=2.2cm,style=multiline,font=\normalfont}

% pretty quotes
\usepackage[autostyle=false, style=english]{csquotes}
\MakeOuterQuote{"}

% fix bug in section headers not showing numbers
\makeatletter
\patchcmd{\ttlh@hang}{\parindent\z@}{\parindent\z@\leavevmode}{}{}
\patchcmd{\ttlh@hang}{\noindent}{}{}{}
\makeatother

% set length of space between paragraphs
\setlength{\parskip}{1em}

% allow breaks in equations
\allowdisplaybreaks

% format section headers
\titleformat{\section}[hang]
  {\normalfont\fontsize{12}{14}\bfseries}
  {}
  {1em}
  {#1}
\titlespacing{\section}{0pt}{1.5em}{0.1em}

\titleformat{\subsection}[hang]
  {\normalfont\fontsize{12}{14}\itshape}
  {}
  {1em}
  {#1}
\titlespacing{\subsection}{0pt}{0.1em}{0.1em}

% disable paragraph indenting
\setlength{\parindent}{0pt}


\begin{document}
\begin{center}

CURRICULUM VITAE

Jeffrey Hanson

School of Biological Sciences, The University of Queensland, \\ St Lucia, Queensland, Australia 4306

\includegraphics[scale=0.2]{envelope-o.png} jeffrey.hanson@uqconnect.edu.au \hspace{0.5em} \includegraphics[scale=0.2]{globe.png} \url{jeffrey-hanson.com}

\end{center}

\section*{RESEARCH INTERESTS}

My research concerns the challenges involved in conserving biodiversity. I am interested in understanding how management actions can achieve conservation objectives for minimal cost. In particular, I specialize in creating cost-effective plans for protected area networks using limited data.

\section*{EDUCATION}
\begin{description}

\item[2013--\\present] PhD, Biology, The University of Queensland, Australia (Advisors: Richard Fuller and Jonathon Rhodes, thesis: \textit{The conservation of evolutionary processes in protected areas})

\item[2011--2012] BS (Hons), First Class, The University of Queensland, Australia (Advisors: Steve Salisbury, Craig Franklin, Hamish Campbell, and Ross Dwyer, thesis: \textit{Using stable isotopes to assess the relationship between body-size, habitat use and diet in estuarine crocodiles (\underline{Crocodylus porosus})})

\item[2007--2010] BS, Major in Ecology, The University of Queensland, Australia

\end{description}

\section*{PROFESSIONAL POSITIONS}
\begin{description}

\item[2012--2013] Research Assistant to Richard Fuller, The University of Queensland, Australia

\item[2012--2013] Research Assistant to Jonathon Rhodes, The University of Queensland, Australia

\item[2012] Casual Professional Staff, The School of Biological Sciences, The University of Queensland, Australia

\end{description}

\clearpage

\section*{PUBLICATIONS}
\subsection*{Journal articles}
\begin{description}

\item[2017] \textbf{Hanson JO}, Rhodes JR, Possingham HP \& Fuller RA raptr: Representative and Adequate Prioritization Toolkit in R. \textit{Methods in Ecology \& Evolution}, In press. DOI: 10.1111/2041-210X.12862.

\item[] \textbf{Hanson JO}, Rhodes JR, Riginos C \& Fuller RA (2017) Environmental and geographic variables are effective surrogates for genetic variation in conservation planning. Under review.

\item[] Mather AT, \textbf{Hanson JO}, Pope LC \& Riginos C (2017) Comparative phylogeography of two co-distributed but ecologically distinct rainbowfishes of far-northern Australia. \textit{Journal of Biogeography}, Accepted.

\item[2016] Dhanjal-Adams KL, \textbf{Hanson JO}, Murray NJ, Phinn SR, Wingate VR, Mustin K, Lee JR, Allan JR, Cappadonna JL, Studds CE, Clemens RS, Roelfsema CM \& Fuller RA (2016) Distribution and protection of intertidal habitats in Australia. \textit{Emu} \textbf{116}: 208--214.

\item[] Dudaniec RY, Worthington Wilmer J, \textbf{Hanson JO}, Warren M, Bell S \& Rhodes JR (2016) Dealing with uncertainty in landscape genetic resistance models: a case of three co-occurring marsupials. \textit{Molecular Ecology}, \textbf{25}: 470-486.

\item[2015] Auerbach NA, Wilson KA, Tulloch AI, Rhodes JR, \textbf{Hanson JO} \& Possingham HP (2015) Effects of threat management interactions on conservation priorities. \textit{Conservation Biology}, \textbf{29}: 1626--1635.

\item[] Bunton JD, Ernst AT, \textbf{Hanson JO}, Beyer HL, Hammill E, Runge CA, Venter O, Possingham HP \& Rhodes JR (2015) Integrated planning of linear infrastructure and conservation offsets. In Weber, T, McPhee, MJ \& Andersson RS (eds) \textit{MODSIM 2015, 21st International Congress on Modelling and Simulation}. Modelling and Simulation Society of Australia and New Zealand, December 2015, pp. 1427--1433.

\item[] \textbf{Hanson JO}, Salisbury SW, Campbell HA, Dwyer RG, Jardine TD \& Franklin CE (2015) Feeding across the food web: The interaction between diet, movement and body size in estuarine crocodiles (\textit{Crocodylus porosus}). \textit{Austral Ecology}, \textbf{40}: 275-286.

\item[] Rabeb D, Othman DS, Essilfie AT, Hansbro PM, \textbf{Hanson JO}, McEwan AG \& Kappler U (2015) Maturation of molybdoenzymes and its influence on the pathogenesis of non-typeable \textit{Haemophilus} influenzae. \textit{Frontiers in Microbiology}, \textbf{6}: 01219.

\item[] Runge CA, Watson JEM, Butchart HM, \textbf{Hanson JO}, Possingham HP \& Fuller RA (2015) Protected areas and global conservation of migratory birds. \textit{Science}, \textbf{350}: 1255-1258.

\end{description}

\subsection*{Popular science articles}
\begin{description}

\item[2015] Beher J \& \textbf{Hanson JO} (2015) Welcome to the Mapotron. \textit{Decision Point}, \textbf{86}: 10--11.

\end{description}

\section*{PEER REVIEW ACTIVITIES}

I have reviewed submissions to following journals: \textit{Austral Ecology},   \textit{Global Change Biology}, and \textit{PLoS ONE}.

\section*{EDUCATIONAL ACTIVITIES}

\subsection*{Classroom Instruction, The University of Queensland, Australia}
\begin{description}

\item[2013--2015] Professional tutor to the "Field Ecology" course, coordinated by Myron Zalucki

\end{description}

\subsection*{Workshop Instruction}
\begin{description}

\item[2017] \textit{Use of Machine Learning in Conservation, Moving beyond just Maxent and SDMs} coordinated by Falk Huettmann at the 28th International Congress for Conservation Biology (ICCB).

\item[2015] \textit{Geospatial Analysis in R} coordinated by Hawthorne Beyer, Rebbecca Runting and Jutta Beher at the Student Conference of Conservation Science, Australia.

\item[] \textit{Smoothing the Marxan Flow with R} coordinated by Matthew Watts at the Student Conference of Conservation Science, Australia.

\item[2013] \textit{Introduction to Geospatial Analysis} coordinated by Hawthorne Beyer at The University of Queensland, Australia.

\item[] \textit{Introduction to Spatial Data Analysis in R} workshop coordinated by Hawthorne Beyer at The University of Queensland, Australia.

\item[2011] \textit{Introducing R} coordinated by Simon Blomberg at The University of Queensland, Australia.

\end{description}

\section*{SCHOLARSHIPS AND AWARDS}
\begin{itemize}

\item Compute resource allocation by the National eResearch Collaboration Tools and Resources (NeCTAR) project (2017)

\item Postgraduate Travel Award Scholarship, The School of Biological Sciences, The University of Queensland, Australia (2016)

\item Australian Postgraduate Award (APA) Scholarship (2013)

\end{itemize}

\section*{PRESENTATIONS}
\subsection*{Conference presentations}
\begin{description}

\item[2016] \textbf{Hanson JO}, Rhodes JR, Possingham HP, Fuller RA (2016) RAPR: Representative and Adequate Prioritizations in R. Oral presentation to Society for Conservation Biology 4th Oceania Congress, Brisbane, Australia.

\item[2014] \textbf{Hanson JO}, Rhodes JR, Fuller RA (2014) Conservation planning for intra-specific biodiversity using surrogates. Oral presentation to the Meeting of the Minds mini-conference at The University of Queensland, Brisbane, Australia.

\end{description}

\subsection*{Seminars}
\begin{description}

\item[2017] \textit{Systematic conservation prioritization in R} presented to members of the Center for Biodiversity and Conservation Science at The University of Queensland, Australia.

\item[2016] \textit{Biodiversity processes in reserve-selection} presented to members of the Center for Biodiversity and Conservation Science at The University of Queensland, Australia.

\end{description}

\subsection*{Scientific meetings and networking events}
\begin{description}

\item[2017] R Unconference hosted by the Brisbane Users of R Group, Brisbane, Australia

\item[2016] rOpenSci Unconference, Brisbane, Australia

\end{description}

\end{document}

\section*{SELECTED PROJECTS}
\begin{description}

\item[raptr] An R package for generating spatial prioritizations that capture within feature variation (eg. intra-specific genetic variation). Available at \url{https://CRAN.R-project.org/package=raptr}.

\item[prioritizr] An R package for creating tailor-made spatial prioritizations. Users can combine a range of different objectives and constraints to find the solution to their problem. Available at \url{https.github.com/prioritizr/prioritizr}.

\end{description}
